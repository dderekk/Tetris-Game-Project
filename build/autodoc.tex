%% Generated by Sphinx.
\def\sphinxdocclass{report}
\documentclass[letterpaper,10pt,english]{sphinxmanual}
\ifdefined\pdfpxdimen
   \let\sphinxpxdimen\pdfpxdimen\else\newdimen\sphinxpxdimen
\fi \sphinxpxdimen=.75bp\relax
\ifdefined\pdfimageresolution
    \pdfimageresolution= \numexpr \dimexpr1in\relax/\sphinxpxdimen\relax
\fi
%% let collapsible pdf bookmarks panel have high depth per default
\PassOptionsToPackage{bookmarksdepth=5}{hyperref}

\PassOptionsToPackage{warn}{textcomp}
\usepackage[utf8]{inputenc}
\ifdefined\DeclareUnicodeCharacter
% support both utf8 and utf8x syntaxes
  \ifdefined\DeclareUnicodeCharacterAsOptional
    \def\sphinxDUC#1{\DeclareUnicodeCharacter{"#1}}
  \else
    \let\sphinxDUC\DeclareUnicodeCharacter
  \fi
  \sphinxDUC{00A0}{\nobreakspace}
  \sphinxDUC{2500}{\sphinxunichar{2500}}
  \sphinxDUC{2502}{\sphinxunichar{2502}}
  \sphinxDUC{2514}{\sphinxunichar{2514}}
  \sphinxDUC{251C}{\sphinxunichar{251C}}
  \sphinxDUC{2572}{\textbackslash}
\fi
\usepackage{cmap}
\usepackage[T1]{fontenc}
\usepackage{amsmath,amssymb,amstext}
\usepackage{babel}



\usepackage{tgtermes}
\usepackage{tgheros}
\renewcommand{\ttdefault}{txtt}



\usepackage[Bjarne]{fncychap}
\usepackage{sphinx}

\fvset{fontsize=auto}
\usepackage{geometry}


% Include hyperref last.
\usepackage{hyperref}
% Fix anchor placement for figures with captions.
\usepackage{hypcap}% it must be loaded after hyperref.
% Set up styles of URL: it should be placed after hyperref.
\urlstyle{same}

\addto\captionsenglish{\renewcommand{\contentsname}{Contents:}}

\usepackage{sphinxmessages}
\setcounter{tocdepth}{1}



\title{autoDoc}
\date{Oct 11, 2023}
\release{2023\_10\_11}
\author{Derek}
\newcommand{\sphinxlogo}{\vbox{}}
\renewcommand{\releasename}{Release}
\makeindex
\begin{document}

\ifdefined\shorthandoff
  \ifnum\catcode`\=\string=\active\shorthandoff{=}\fi
  \ifnum\catcode`\"=\active\shorthandoff{"}\fi
\fi

\pagestyle{empty}
\sphinxmaketitle
\pagestyle{plain}
\sphinxtableofcontents
\pagestyle{normal}
\phantomsection\label{\detokenize{index::doc}}



\chapter{Indices and tables}
\label{\detokenize{index:indices-and-tables}}\begin{itemize}
\item {} 
\sphinxAtStartPar
\DUrole{xref,std,std-ref}{genindex}

\item {} 
\sphinxAtStartPar
\DUrole{xref,std,std-ref}{modindex}

\item {} 
\sphinxAtStartPar
\DUrole{xref,std,std-ref}{search}

\end{itemize}


\chapter{TetrisAI Class}
\label{\detokenize{index:tetrisai-class}}
\sphinxAtStartPar
The \sphinxtitleref{TetrisAI} class is the AI component of the Tetris game, designed to predict the best cube moves and rotations to achieve the highest scores.


\section{Methods}
\label{\detokenize{index:methods}}\begin{description}
\sphinxlineitem{\sphinxtitleref{calculate\_aggregate\_height(self, background, original\_data)}}
\sphinxAtStartPar
Calculate the aggregate height difference between the background and the original data.

\sphinxlineitem{\sphinxtitleref{calculate\_height\_difference(self, background)}}
\sphinxAtStartPar
Calculate the height difference between the highest and lowest points of the background.

\sphinxlineitem{\sphinxtitleref{calculate\_complete\_lines(self, background)}}
\sphinxAtStartPar
Calculate the number of complete lines in the background.

\sphinxlineitem{\sphinxtitleref{calculate\_holes(self, background)}}
\sphinxAtStartPar
Calculate the number of holes in the background.

\sphinxlineitem{\sphinxtitleref{calculate\_bumpiness(self, background)}}
\sphinxAtStartPar
Calculate the bumpiness of the background.

\sphinxlineitem{\sphinxtitleref{calculate\_well\_sums(self, background)}}
\sphinxAtStartPar
Calculate the well sums in the background.

\sphinxlineitem{\sphinxtitleref{calculate\_row\_transitions(self, background)}}
\sphinxAtStartPar
Calculate the number of row transitions in the background.

\sphinxlineitem{\sphinxtitleref{calculate\_column\_transitions(self, background)}}
\sphinxAtStartPar
Calculate the number of column transitions in the background.

\sphinxlineitem{\sphinxtitleref{calculate\_score(self, background, original\_data)}}
\sphinxAtStartPar
Calculate the score for a given board state using the El\sphinxhyphen{}Tetris algorithm.

\sphinxlineitem{\sphinxtitleref{predict\_best\_move(self, background, block\_ini\_position, current\_block\_shape)}}
\sphinxAtStartPar
Predict the best move (rotation, left/right movement, and downward movement) for the current block on the background.

\end{description}


\section{Attributes}
\label{\detokenize{index:attributes}}\begin{description}
\sphinxlineitem{\sphinxtitleref{WEIGHTS}}
\sphinxAtStartPar
A dictionary containing weights for each feature used in scoring.

\end{description}


\section{Examples}
\label{\detokenize{index:examples}}
\sphinxAtStartPar
Here are some examples of how to use the \sphinxtitleref{TetrisAI} class:

\sphinxAtStartPar
{\color{red}\bfseries{}\textasciigrave{}\textasciigrave{}}{\color{red}\bfseries{}\textasciigrave{}}python
\# Create a TetrisAI instance
ai = TetrisAI(block\_data, block\_control)

\sphinxAtStartPar
\# Predict the best move for the current block
best\_move = ai.predict\_best\_move(background, block\_ini\_position, current\_block\_shape)

\sphinxAtStartPar
\# Calculate the score for the current board state
score = ai.calculate\_score(background, original\_data)



\renewcommand{\indexname}{Index}
\printindex
\end{document}